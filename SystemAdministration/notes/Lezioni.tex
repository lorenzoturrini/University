\documentclass[12pt]{article}

\title{LAS - Appunti lezioni}
\author{Daniel}
\date{\today{}}

\usepackage{graphicx}
\usepackage[margin=1in]{geometry}
\usepackage{float}
\usepackage{wrapfig}

\begin{document}
	
	\maketitle
	\newpage
	
	\section*{Appunti lezioni}
		\subsection{3/3/20}
		Pacchi di slide viste:
		\begin{itemize}
			\item lab-virtualbox.pdf
			\item interfaccia di testo
		\end{itemize}




			\paragraph{Setup Macchine Virtuali}
			La reinizializzazione dei macaddres e' indispensabile nel caso in cui le macchine virtuali debbano coesistere nello
			stesso segmento di rete. In caso contrario non e' strettamente necessario. Nel file /etc/udev/rules.d/70-persistent-net.rules
			e' possibile trovare configurazioni per la persistenza delle assegnazioni $interfaccia-rete \iff mac-addres.$
			\\	
			Quando si passa ad un altro utente si puo' usare il comando su, consigliato con il trattino per importare e settare tutto
			l'ambiente di un eventuale utente (var di ambiente, etc etc)
			per tornare all'utente di partenza CTRL-D ( o exit)
			per fare clean della cli (CRTL-L) o clean
			CRTL-S per freezare il terminal - CRTL-Q per sbloccarlo
			\\
			SUDO: quando eseguo sudo l'ambiente di esecuzione rimane invariato a quello del chiamante.
				perche' sudo ls -l /root > /root/prova non funziona?
				perche' sudo agisce solo sul comando ls, quando la shell prova ad aprire in scrittura il file /root/prova ovviamente
				fallisce perche' las (utente da cui si e' lanciata la shell) non ha i privilegi necessari.
				sudo -i apre una sessione interattiva con i privilegi di root, in questa maniera l'intera gerarchia dei
				processi che lancio dalla shell verra' lanciata come root e quindi da utente privilegiato. (VEDERE MAN PAGE SUDO)
				L'utente las puo' utilizzare sudo digitando la propria passwd, questa richiesta e' un layer di sicurezza di secondo ordine
					- l'utente si allontana lasciando il terminale incustodito e un malintezionato prova a inserire comandi
					- risvegliare l'attenzione dell'utente prima di svolgere un job da root (cerca di qualcosa con privilegi elevati)
				Sudo inolte possiede un sistema di caching interno, una volta inserita la passwd , questa viene tenuta in cache per 5 min
				in modo da non doverla ridigitare continuamente.
				Comando visudo per modificare /etc/sudoers contenente le configurazioni di sudo e i permessi assegnati.
				%sudo ALL=(ALL:ALL) ALL questa riga puo' essere modificata ulteriormente per poter diventare root con sudo senza
				dover inserire la passwd. (TODO:HOW?)
			\\
			/etc/hosts permette di risolvere localmente richieste DNS
			\\
			per disattivare un'interfaccia basta commmentare le righe relative nel file /etc/network/interfaces
			\\
			CONFIGURARE /etc/netowork/interfaces per macchine client router server
			\\
			esistono tool per utilizzare il mouse in cli , utile essenzialmente per fare operazioni copia incolla. 

			

			\paragraph{CLI}
			interfaccia al sistema operativo. Interprete di comandi. Il linguaggio della shell si puo' pensare come un linguaggio
			general purpose, ha tutte le caratteristiche che servono per poter scrivere qualsiasi tipo di algoritmo.
			La shell offre un linguaggio che permette di automatizzare job, eseguire automaticamente set di operazioni

			4 macrocategorie di comandi
			
			BASH Manual

			Codici ANSI


			La shell deve distinguere sulla riga le diverse parti che successivamente all'espansione divenetaranno gli argomenti.

			suffisso 'rc' = resource configuration

				- comando touch

			




	\subsection{hardening e controllo dell'accesso}
		\paragraph{Sicurezza} la sicurezza e' il risultato di un processo, che tiene in considerazione tutti gli aspetti non solo tencologici
		ma anche umani. Alcuni elementi organizzativi che possono tornare utili:
		- in generale la sicurezza si ottiene quando tutti gli elementi presenti posso agire in accordo alla politica dei privilegi minimi
		- la sicurezza e' antagonista dell'usabilita'
		si cercano soluzioni che pongano meno ostacoli possibili fra il servizio in oggetto e gli utenti finali che vogliono utilizzarlo

		\paragraph{messa in sicurezza fisica}
		la predisposizione fisica ha una sua importanza rilevante.

		- collocazione hardware
		- allocazione delle risorse
		- procedura di avvio del sistema
		
		TODO:
			- guardare MAN page dei tools:
				-adduser
				-addgroup
				-chown
			- concetto di entropia

	

	\section{Laboratorio}
	
	
	
\end{document}